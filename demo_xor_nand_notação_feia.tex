\documentclass{article}
\usepackage{amsmath}
\usepackage{booktabs} % for "\midrule" macro
%\usepackage{lipsum} % for filler text
%\usepackage{caption}
%\newcommand{\xor}{\oplus}
\newcommand{\xor}{%
  \mathbin{%
    {\vee}\mspace{-2.9mu}\nonscript\mspace{0.3mu}{\vee}%
  }%
}
\makeatletter
\DeclareRobustCommand{\nand}{\mathbin{\mathpalette\n@and@or\land}}
\DeclareRobustCommand{\nor}{\mathbin{\mathpalette\n@and@or\lor}}

\newcommand{\n@and@or}[2]{%
  \vphantom{#2}%
  \ooalign{$\m@th#1#2$\cr\hidewidth$\m@th#1\sim$\hidewidth\cr}%
}
\makeatother



\begin{document}
\title{Demonstração XOR usando apenas NANDs}
\author{Autor e vai e vai e vai}
\maketitle
\section{Tabela verdade de uma NAND}
A NAND B pode ser escrito como $\lnot(A\land B)$. Abaixo sua tabela verdade:
\begin{center}
    Tabela Verdade $\lnot (A\land B)$
\end{center}
%\bigskip % some extra vertical space *above* the first "center" env.
\begin{center}
\begin{tabular}{cccc}
$A$ & $B$ & $A\land B$ & $\lnot (A\land B)$\\
\midrule
V & V & V & F\\
V & F & F & V\\
F & V & F & V\\
F & F & F & V\\
\end{tabular}
\end{center}
Como verificado acima, uma NAND funciona como uma AND (um ``e'') mas com sua saída (resultado) negado. Agora seguiremos para uma tabela verdade de uma XOR (ou exclusivo).
\section{Tabela verdade de uma XOR}
Podemos escrever A XOR B como $A\oplus B$ ou como $(\lnot A\land B)\lor (A\land \lnot B)$. Verificamos a tabela abaixo:
\begin{center}
   Tabela Verdade $A \oplus B$ 
\end{center}
%\bigskip % some extra vertical space *above* the first "center" env.
\begin{center}
\begin{tabular}{ccccccc}
$A$ & $\lnot A$ & $B$ & $\lnot B$ & $\lnot A\land B$ & $A\land  \lnot B$ & $\smash{\overbrace{(\lnot A\land B)\lor (A\land \lnot B)}^{\textbf{($A \oplus B$)}}}$\\
\midrule
V & F & V & F & F & F & F\\
V & F & F & V & F & V & V\\
F & V & V & F & V & F & V\\
F & V & F & V & F & F & F\\
\end{tabular}
\end{center}
\subsection{O problema}
Para demonstrar que é possível construir uma proposição lógica equivalente a um ``ou exclusivo'' (XOR), mas usando somente ``nãos'' (NOTs) e ``es'' (ANDs), ou seja NANDs, precisaríamos transformar a proposição inicial, removendo o ``ou'' (OR) com algum artifício lógico disponível.
\subsubsection{De Morgan e dupla negação}
A primeira forma de remover o ``ou'' (OR) seria usando ``De Morgan'', que descreve que:
\begin{center}
$\lnot (A\land B) \equiv \lnot A \lor \lnot B$   
\end{center}
e também que:
\begin{center}
$\lnot (A \lor B)  \equiv \lnot A \land \lnot B$   
\end{center}
Mas como fica visível, precisaríamos que as partes da proposição estivessem ambas negadas ou que a própria proposição inteira estivesse negada.
Podemos então desenvolver $(\lnot A\land B)\lor (A \land \lnot B)$ usando uma dupla negação:
\begin{center}
    $(\lnot A\land B)\lor (A\land \lnot B) = \lnot\lnot ((\lnot A\land B)\lor (A\land \lnot B))$
\end{center}
Agora, a partir de \emph{De Morgan}, temos que:
\begin{center}
  $\lnot(\lnot((\lnot A\land B)\lor (A\land \lnot B))) = \lnot(\lnot(\lnot A\land B)\land \lnot(A\land \lnot B))$  
\end{center}
\subsubsection{O problema de ¬A e ¬B}
Da forma atual, a expressão tem ``nãos isolados'', nos casos de \textbf{A} e \textbf{B}, e então precisamos desenvolver $\lnot A$ e $\lnot B$. Logo, como verificado na tabela:
\begin{center}
\begin{tabular}{cccccccc}
$A$ & $\lnot A$ & $A\land A$ & $\lnot (A\land A)$ & $B$ & $\lnot B$ & $B\land B$ & $\lnot (B\land B)$\\
\midrule
V & F & V & F & V & F & V & F\\
V & F & V & F & F & V & F & V\\
F & V & F & V & V & F & V & F\\
F & V & F & V & F & V & F & V
\end{tabular}
\end{center}

Sabemos que:

\begin{center}
    $\lnot A = \lnot (A\land A)$ e $\lnot B = \lnot (B\land B)$,
\end{center}
 
%\bigskip  % some extra vertical space *below* the second "center" env.

Agora temos que:

\begin{center}
    $\lnot(\lnot(\lnot A\land B) \land \lnot(A\land \lnot B)) = \lnot(\lnot(\lnot(A\land A)\land B)\land \lnot(A\land \lnot(B\land B)))$
\end{center}
\subsection{Desenvolvimento da tabela verdade}
Agora complementando a tabela, e usando os resultados da tabela anterior, temos:

\begin{center}
   Tabela Verdade Final - Parte 1 
\end{center}

\begin{center}
\begin{tabular}{cccc}
$\lnot (A\land A)\land B$ & $\lnot(\lnot(A\land A)\land B)$ & $A\land \lnot(B\land B)$ & $\lnot(A\land \lnot(B\land B))$\\
\midrule
F & V & F & V\\
F & V & V & F\\
V & F & F & V\\
F & V & F & V 
\end{tabular}
\end{center}
\bigskip

E em seguida construindo a tabela para a conjunção dos termos negados da tabela anterior:
\begin{center}
   Tabela Verdade Final - Parte 2 
\end{center}

\begin{center}
\begin{tabular}{ccc}
$\lnot(\lnot(A\land A)\land B)$ & $\lnot(A\land \lnot(B\land B))$ & $\lnot(\lnot(A\land A)\land B)\land \lnot(A\land \lnot(B\land B))$\\
\midrule
V & V & V\\
V & F & F\\
F & V & F\\
V & V & V
\end{tabular}
\end{center}
\bigskip

E agora a negação da conjunção:
\begin{center}
   Tabela Verdade Final - Parte 3 
\end{center}
\begin{center}
\begin{tabular}{cc}
$\lnot(\lnot(A\land A)\land B)\land \lnot(A\land \lnot(B\land B))$ & $\lnot(\lnot(\lnot(A\land A)\land B)\land \lnot(A\land \lnot(B\land B)))$\\
\midrule
V & F\\
F & V\\
F & V\\
V & F\\
\end{tabular}
\end{center}
\bigskip

E finalmente, comparando com os resultados da tabela inicial:
\bigskip
\bigskip

%\begin{center}
%   Tabela Verdade Final
%\end{center}

\begin{center}
\begin{tabular}{cc}
$\lnot(\lnot(\lnot(A\land A)\land B)\land \lnot(A\land \lnot(B\land B)))$ & $\smash{\overbrace{(\lnot A\land B)\lor (A\land \lnot B)}^{\textbf{($A \oplus B$)}}}$\\
\midrule
F & F\\
V & V\\
V & V\\
F & F\\
\end{tabular}
\end{center}
\bigskip
\section{Conclusão}
Verificando que as duas tabelas têm as mesmas saídas (continuam a ter, na verdade), podemos concluir então, que é possível descrever $A\oplus B$ (A XOR B) usando apenas NANDs.

Escrevendo como no início do problema, em que A NAND B = $\lnot(A\land B)$, teríamos que:

\begin{center}
 $A \oplus B = \lnot(\lnot(\lnot(A\land A)\land B \land(\lnot(A\land(\lnot(B\land B))))))$   
\end{center}

pode ser escrito também como:
\begin{center}
    A XOR B = ((A NAND A)NAND B) NAND (A NAND(B NAND B))
\end{center}



ou com outra notação:

\begin{center}
    $A \xor B = ((A\nand A)\nand B)\nand(A\nand(B\nand B))$
\end{center}   
\bigskip


ou com mais uma outra notação:


\begin{center}
  $A \oplus B = \overline{\overline{(\overline{(AA)}.B)}.\overline{(A.\overline{(BB)})}}$  
\end{center}


\end{document}